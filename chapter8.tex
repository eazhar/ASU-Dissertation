\chapter{Conclusion}

In this dissertation, the integration of photovoltaic and electrochromic devices for self-powered smart window applications has been investigated.  The design constraints for realizing such an integrated device may significantly interact with one another, compounding its complexity.  For a photovoltaic device, high transparency yields reduced utilization of the terrestrial solar spectrum, and electrochromic films should be chosen such that the power requirements for initiating the reduction-oxidation reaction are minimal.  This should be considered in conjunction with low transition (coloration and bleaching) times, and high optical density.  The focus of this dissertation has been toward investigating three classes of solar cells (inorganic, hybrid organic, and fully organic) devices toward minimally powering electrochromic optimized for low transition voltages.

Inorganic Zinc Oxide (ZnO) thin film-based metal-semiconductor (MS), metal-insulator-semiconductor (MIS), and p-i-n heterojunction devices were investigated, and their photovoltaic conversion under ultraviolet (UV) illumination was evaluated.  For MS devices, photoconductive behavior under ultraviolet illumination ($\lambda$=365 nm), suggesting the outsized role of surface states.  Fitted against the generalized Bardeen model, Schottky parameters were used to estimate the density of interface states (D$_{it}$ $\approx$ 8.0$\times$10$^{11}$ eV$^{-1}$cm$^{-2}$) and the neutral level (E$_{o}$ $\approx$ -5.2 eV).  Post-metalization oxygen plasma treatment of Ag and Ti/Ag electrodes resulted in a net Schottky barrier height increase; linked to the formation of Ag$_{2}$O and TiO$_{x}$. The effective barrier potential maximized with a 20 \AA{} electron blocking layer (EBL, SiO$_{2}$), suppressing recombination at the metal/semiconductor interface and blocking majority carrier current flow. Photovoltaic performance of p-i-n heterojunction structures was maximized generating an open-circuit voltage (V$_{oc}$) of 1.2 V and short circuit current density (J$_{sc}$) of 0.68 mA/cm$^2$ for interdigitated devices under high energy monochromatic UV-C radiation. When properly scaled, ZnO thin film absorbers with sufficiently thin EBL and high surface barrier electrodes are suitable for visibly transparent, low-power smart-windows.

A fully organic photovoltaic, composed of poly[N-9’-heptadecanyl-2,7-carbazole-alt-5,5-(4’,7’-di-2-thienyli2’,1’,3’-benzothiadiazole)]:phenyl-C71-butyric-acid-methyl \\(PCDTBT:PC70BM) was explored the final alternative approach toward powering electrochromic devices. The bulk heterojunction exhibited high transparency and relatively large power conversion efficiency and provided the requisite power for transitioning an inkjet printed, nano-particle-based, tungsten trioxide (WO$_{3}$) electrochromic films on substrates of varying mechanical flexibility. The printed electrochromic devices demonstrated clear switching behavior under external bias, with 7 second coloration time, 8 second bleaching time, and 0.36-0.75 optical modulation at $\lambda$=525 nm.   The bulk heterojunction devices were evaluated with varying hole-transport layers and substrates, and exhibited the strongest performance of PCE$\approx{}$3\%, V$_{oc}$=0.9V and J$_{sc}\approx{}$10-15 mA/cm$^{^2}$.  The devices were paired using an Internet of Things controller enabling wireless switching.

The purview of this dissertation focuses on the pairing of photovoltaic and electrochromic devices.  The realization of such an integrated smart window presents various challenges that require tradeoffs among device design, selection of electrical and optical materials, and reproducibility/reliability of low-temperature processing. Here, we focused on the processing and performance of three classes of visibly-transparent photovoltaic materials.  This interdisciplinary research approach, allowing for the reconciliation of multiple goals and trade-offs within a single design problem.  The stages of planning, executing, and analysis requires extensive due diligence in addition to confirming abstract expectations realized in the physical world.  The broader impact of this work may affect household and building envelope energy consumption with the novel integrated device presented.  Furthermore, the use of organic and nanomaterials may accelerate the pace of manufacturability of the burgeoning field of self-powered electrochromic devices.