\chapter{Synthesis and Fabrication of ZnTe Nanosheet Field Effect Transistors and Photodetectors}

\section{Abstract}
In this work, we report the fabrication of ZnTe nanosheet devices grown by a high temperature growth process followed by low temperature fabrication process.  Single crystal ZnTe nanosheets were grown on Si by heating ZnTe powder in a tube furnace, promoting the synthesis of high quality of ZnTe nanosheets. Scanning electron microscopy, atomic force microscopy, and transmission electron microscopy indicate ordered and single crystalline sheets have been obtained.  Transistor devices were then fabricated with the nanostructures.  The current-voltage characteristics were used to study the transport behavior of single ZnTe nanowires and nanosheets.  Predominant p-type behavior was observed, an on/off ratio of 3.7 $\times$ 10$^{3}$ was found, and a field effect mobility of 4.11 cm$^{2}$/V-s.  The photoconductivity of these nanosheets was also investigated under the illumination of a green diode laser with varying intensity.  A responsivity of 2.62 $\times$ 10$^{5}$ A/W and a photoconductive gain of 6.10 $\times$ 10$^{5}$  was determined for nanowires, as well as 110.59 A/W and a photoconductive gain of 257.76 for nanosheets.  These values are quite comparable to similar ZnTe nanodetector devices.  The mechanism for photoconductive gain in ZnTe has largely been attributed to surface states creating trap recombination centers leading to longer carrier lifetimes.  These results reveal that single ZnTe nanowires are excellent candidates for applications for nanoscale photoconductive detectors.
\section{Introduction}
Single and two dimensional nanostructures have garnered much interest in regards to their realization toward functional devices \citep{lieber_functional_2007, agarwal_semiconductor_2006, wang_zno_2009}.  Two-dimensional nanostructures possesses very unique advantages by nature of their geometry, including mesoscopic assembly, planar surfaces, and the ability to be utilized as miniature substrates \citep{park_scaffolding_2004, zhang_single-crystal_2011}.  Synthesis techniques of such nanostructures, including nanowires and nanosheets, have been widely studied, and these structures have been utilized for a variety of applications including sensors \citep{ramgir_nanowire-based_2010, belagodu_modulation_2012,cui_nanowire_2001},  field effect transistors \citep{xu_charge_2012, cui_high_2003}, photovoltaics \citep{tian_single_2009, tian_coaxial_2007}, and photodetectors \citep{yu_zno_2012, young_enhanced_2015, soci_zno_2007}.  
Certain interest has been placed on Zinc Telluride, a II-VI compound semiconductor, with a band gap of 2.26 eV and exhibits preferential p-type doping \citep{ruda_compensation_1991, larach_anomalous_1957}. Doped and undoped ZnTe films have traditionally been used to realize a variety of optoelectronic devices such as photodetectors \citep{chang_aluminum-doped_2001} and intermediate solar cell layers \citep{gessert_dependence_2009, rioux_znte:_1993, spath_studies_2005}.  Additionally, ZnTe has often been compared to GaAs and ZnS systems in regards to its superior electro-optical coefficient and electron-LO-phonon (polaron) coupling constant \citep{kumagai_growth_2012, pelekanos_hot-exciton_1991, kaminow_electrooptic_1966, xue_bond-charge_1997}, as well as GaN/InGaN emitters for circumventing the problem of phase segregation in compositional band edge engineering \citep{osamura_preparation_1975, moon_effects_2001}.  Most recently, the growth mechanisms describing the formation of various ZnTe nanostructures have been heavily explored \citep{devami_synthesis_2011, li_fabrication_2005, janik_catalytic_2007, meng_temperature-dependent_2008, guo_fabrication_2008,yong_formation_2007, you-wen_electrochemical_2007}.  Additionally the use of ZnTe nanowires has led to the successful fabrication of field effect transistors \citep{cao_-situ_2012,huo_electrical_2006,li_enhanced_2010,li_structure_2011,liu_fabrication_2013,wu_device_2012}  and photodetectors \citep{zhang_p-type_2008, liu_fabrication_2013, li_structure_2011}.   However, the this discussion and an adequate comparison have been scarce among ZnTe structures exhibiting a unique, 2D nanosheet configuration.
In this study, ZnTe nanowires and nanosheets have been synthesized and characterized.  Their relative performance in terms of electrical conductivity, transfer behavior, and photoresponse have been compared.  It has been found that the pronounced effect of surface trap states render unfavorable transfer behavior for nanowires, but contribute to the greater photodetection sensitivity, relative to nanosheets.
\section{Experimental Details}
Zinc Telluride (ZnTe) nanosheets and nanowires were synthesized in a quartz tube placed in a single zone horizontal tube furnace. Si substrates [100] were cleaned successively with acetone, isopropanol, and deionized water. Nanostructure synthesis was facilitated by a VLS growth technique \citep{huo_synthesis_2006, meng_temperature-dependent_2008, devami_synthesis_2011}. ZnTe (99.99\%) powder was placed at the center of the quartz tube and Au-coated (30 nm sputtered) silicon substrates were placed at downstream from the source material. 

After the quartz tube was evacuated, the samples were then held at 100 \degree C overnight before commencing synthesis.  High purity argon was  introduced at a rate of 140 sccm (standard cubic centimeters per minute) into the quartz tube. The powder was elevated to a temperature of 750 \degree C and the temperature of the substrates were maintained at approximately 570 \degree C for the nanosheets and at 475-530 \degree C for the nanowires. A pressure of 20 torr was maintained within the tube furnace throughout the entirety of synthesis. After a growth time of 5 hours, the furnace was allowed to cool naturally to room temperature, while maintaining the above mentioned flow rate and pressure conditions. Based on the above growth parameters, both nanosheets and nanowires were formed. The samples were further characterized with a field emission scanning electron microscope (SEM, Hitatchi S4700-II), operated at 15 kV, a  atomic force microscope (Digital Instruments) for profile analysis, and a transmission electron microscope (TEM, JEOL JEM 2010F) attached with an energy-dispersive X-ray spectrometer (EDAX). 

Assembly of single ZnTe nanosheet and nanowire-based devices was accomplished with a standard microelectronic-fabrication process.  Transfer characteristics of the FET and the current-voltage (I-V) characteristics of the photodetectors were measured using a probe station connected to a Keithley 4200 Semiconductor Characterization System (SCS).  Devices were then illuminated under a 532 nm, power modulated, green diode laser.  All electrical measurements were collected at room temperature in air.

\begin{figure*}[t] %figure* makes it take up the whole width and move the text it would run into as a result, [t] puts it at the top of the next page
\centering %centers the picture
\includegraphics[width=\textwidth]{figures/fig3_1}%specify that the figure takes up the width of the page
\caption{(a) SEM micrograph of synthesized ZnTe nanostructures. Nanosheets are observed with the presence of a single nanowire protruding and establishing the foundation of the nanosheet base. (b) Illustration of growth and filling process forming ZnTe nanosheets (c) AFM and profile analysis of single ZnTe nanosheet}
\label{fig:fig3_1}
\end{figure*}

%%Figure 2 Defined Here%%
\begin{figure}[b]
\centering %centers the picture
\includegraphics[scale=0.5]{figures/fig3_2}
\caption{EDX Spectrum of ZnTe nanosheet (a) HRTEM Micrograph of ZnTe nanosheet illustrating the (111) plane spacing of 0.349 nm with (b) corresponding SAED Pattern
}
\label{fig:fig3_2}
\end{figure}
%%Figure 2 Defined Here%%

\section{Results and Discussion}
Figure \ref{fig:fig3_1}(a) shows the scanning electron micrograph (SEM) of various high quality single crystal ZnTe nanostructures as confirmed by previous XRD analysis \hl{[Cite Pengs growth paper]}. The ZnTe nanostructures are found to be in different stages of growth, in addition to dispersed Te precipitates (confirmed with EDX). Measurements from the microcraph reveal that nanowire length ranges from tens to hundreds of microns that their diameters range from 90-230 nm.  Very distinct triangular growth structures, nanosheets, have also been found throughout the synthesis.  AFM (Figure \ref{fig:fig3_1}(c)) analysis confirms the thickness of the ZnTe nanosheets to be on the order of 100-500 nm. Evidence of nanowire artifacts are seen throughout each ZnTe nanosheet, as they seem to be present along the base of each nanosheet formation.  

%%Figure 3 Defined Here%%
\begin{figure}[h]
\centering %centers the picture
\includegraphics[scale=0.60]{figures/fig3_3}
\caption{(a) Schematic of ZnTe nanostructure devices and (b) SEM micrograph of single nanosheet device}
\label{fig:fig3_3}
\end{figure}
%%Figure 3 Defined Here%%



Figure \ref{fig:fig3_1}(b) illustrates the proposed growth formation of ZnTe nanosheets. As discussed in reports of ZnO nanosheet growth \citep{park_ultrawide_2004, park_synthesis_2005}, gold catalyzed VLS growth has been shown in several cases to lead to secondary dendritic side branched growth. In the final stages of formation, the dendrites will tend to fill the interspaces monolithically.  This growth is proposed to occur on the Te-terminated surface in the \textless 111\textgreater{} family of directions because this face has the lowest surface energy, and higher growth rate than the Zn-terminated surface. The low energy surfaces grow to decrease the overall surface energy of the crystal.  

Previous studies of ZnTe nanostructures growth offer multiple explanations in regards to planar nanostrucutred ZnTe formation, but not specifically large-scale nanosheets as described above.  One common thread, however is that high temperature and high vapor pressure lead to a more diffusion-limited growth, which promotes secondary formation. Utama et. al proposes that nanobelts are formed from 2D growth at high temperatures from VS in the lateral direction in addition to axial growth from VLS. This 2D growth has a pronounced tapering associated with the structure. In regards to the twinned nanobelts, it has also been found that the Te-terminated facet is more chemically and catalytically active than Zn, making it an active facet for lateral crystal growth \citep{utama_growth_2013}. Devami et. al also note that the high temperature ribbon/belt formation is from the 2D epitaxy concurrent with axially growing wires. A combination of increased flow rate, and higher temperatures combined with rate-limited precursor, the nanowires formed are short and thin in nature while other structures form such as multiple prongs on wire tips. With an increase in temperature these multi-pronged wires taper and with an increase in temperature and nanoribbons begin to form \citep{devami_synthesis_2011}. 

\begin{figure*}[t]
\centering %centers the picture
\includegraphics[scale=0.60]{figures/fig3_4}
\caption{Electrical characteristics of nanowire (first column) and nanosheet devices (second column).  (a) and (b) I-V curves of nanowire and nanosheet devices under zero gate-bias and no illumination for contact barrier analysis.  (c) and (d) I$_{ds}$ vs. V$_{ds}$ curves of nanowire and nanosheet devices at varied V$_{gs}$.  Note: Reduced conductivity as V$_{gs}$ increases indicating p-type behavior (inset shows optical micrograph of both devices) (e) and (f) I$_{ds}$ vs. V$_{gs}$ curves for nanowire and nanosheet devices, respectively.}
\label{fig:fig3_4}
\end{figure*}

Figure \ref{fig:fig3_2}(a) shows the high resolution TEM (HRTEM) micrograph of a ZnTe nanosheet, which confirm a single-crystalline lattice structure. The ZnTe spacings were extracted by a fast fourier transform (FFT) analysis. The zinc-blende ZnTe structure with a lattice spacing of 0.349 nm which corresponds to the d-spacing of the (111) plane of the ZnTe structure. Additionally, the lattice spacing of d=0.3 nm corresponds to the (200) plane of the zinc blende ZnTe structure.  An SAED pattern is also shown in Figure \ref{fig:fig3_2}(b) which indexes the [111] and [220] directions. An EDX spectrum is also found in Figure \ref{fig:fig3_2} which confirms the nanosheet composition of Zn and Te.  The presence of the Cu signal arises from the Cu TEM grid.

Both nanostructures were then fabricated into transistor devices through a conventional photolithographic patterning process on a heavily doped p+ Si substrate, with a 200 nm thermally grown SiO$_2$ back-gated insulator.  A layer of Cr/Au (10 nm / 250 nm) was thermally evaporated, forming top contacts.  The schematic of this device structure is seen in the inset of Figure \ref{fig:fig3_3}(a).  The fabricated devices were formed with 5 $\mu$m channel lengths extended between electrodes.  An optical micrograph of both device types can be found in the SEM of the nanosheet device found in Figure \ref{fig:fig3_3}(b).

The current-voltage (I-V) response of both ZnTe nanowire and nanosheet devices with no gate bias and in dark conditions is found in Figure \ref{fig:fig3_4} (a) and (b).  Both devices displayed slight nonlinearity and asymmetry based on the polarity of the applied voltage.  This indicates the presence of a contact barrier between Cr/Au and ZnTe, and fitting the 0 gate-bias plot to a thermionic emission model, we are able to extract the Schottky barrier height ($\phi_B$) and ideality factor (n) according to the following relationships \citep{sze_physics_2006}:

$\phi_B$ = ${\frac{kT}{q}}ln(\frac{A^{*}T^2}{J_o})$

n = $\frac{q}{kT}\frac{dV}{d(ln(J))}$

where k is the boltzmann constant, T is the absolute temperature (taken at room temperature), $A^{*}$ is the effective Richardson constant (taken to be 72 $\frac{A}{cm^{2}K^{2}}$) \citep{bhunia_schottky_2000}, $J_{o}$ is the extrapolated current density intercept, q is the elementary electronic charge and dV/d(ln(J)) is the inverse of the slope of the semi-logrithmic I-V plot. In this case, a barrier height of 0.48 eV with an ideality factor of 0.50 was determined for nanowires and 0.69 eV with an ideality factor of 0.74 nanosheets.  Little has been reported specifically of Schottky barriers on ZnTe nanostructures formed with Cr/Au.  However reports of Schottky barriers formed with Au on bulk ZnTe\citep{baker_schottky_19a} have demonstrated barrier heights of 0.69, and 0.44 eV with Ti/Au on ZnTe quantum dots \citep{zielony_electro-optical_2012}, while others have reported ohmic behavior\citep{meng_ohmic_2009}.  In this work, the increased conductivity of the nanowire, due to surface dominated transport and self-compensation), gives way to an increased current density and increased inverse-slope (due to increased SRH recombination), thereby reducing the Schottky barrier height and ideality factor, as compared to the planar sheet configuration.

\begin{table*}[htbp]
%p{5cm} is the space given to wrap the text
\begin{tabular}{l | c | c | c | c}
Device Structure & g$_{m} (nS)$ & \mu$_{FE}$ cm$^{2}$/V-s & I$_{on}$/I$_{off}$ & S (mV/dec)\\
\hline \hline
\parbox[t]{4cm}{ZnTe Bulk \\ \citep{fujita_electrical_1975}} & n.a. & 31 (hall) & n.a. & n.a. & 
\parbox[t]{4cm}{ZnTe NW \\ \citep{liu_fabrication_2013}} & 488.5& 11.3&  3.3 $\times$ 10$^{3}$ &  n.a. &
\parbox[t]{4cm}{ZnTe NW \\ \citep{cao_-situ_2012}} &  n.a. &  1.76 $\times$ $10^{-4}$ & 5&  n.a. &
\parbox[t]{4cm}{ZnTe NW (twinned) \\ \citep{li_structure_2011}} & 0.34& 0.11&  ~1.3 &  n.a. &
\parbox[t]{4cm}{ZnTe:Cu NW \\ \citep{huo_electrical_2006}} & 1.7& 1& n.a. &  n.a. &
\parbox[t]{4cm}{ZnTe:Ph (.5 at \%) NW \\ \citep{cao_-situ_2012}} &  n.a. &  4.47 $\times$ $10^{-3}$ & n.a. &  n.a. &
\parbox[t]{4cm}{ZnTe:Sb nanoribbons \\ \citep{wu_device_2012}} & 8.1& 0.94&  ~1.6 &  2.94 $\times$ 10$^{5}$  &
\parbox[t]{4cm}{ZnTe: nanoribbons \\ \citep{li_enhanced_2010}} & 0.4& 0.01& n.a. &  n.a. &
\parbox[t]{4cm}{ZnTe:N: nanoribbons \\ \citep{li_enhanced_2010}} & 78.4& 1.2& n.a. &  n.a. &
\parbox[t]{4cm}{ZnTe NW \\ (this work)} & 1.5& 1.51&  3.16 $\times$ 10$^{3}$ &  4.47 $\times$ 10$^{3}$ &
\parbox[t]{4cm}{ZnTe nanosheet \\ (this work)} & 212.7& 4.11&  3.75 $\times$ 10$^{3}$ &  2.42 $\times$ 10$^{4}$ &
\end{tabular}
\caption{Comparison of ZnTe nanostructure-based transistor performance metrics}
\label{tab:tbl3_1}
\end{table*}

A family of I$_{ds}$ vs. V$_{ds}$ curves at various gate voltages is also shown for a single nanowire (Figure \ref{fig:fig3_4}(c)) and a single nanosheet (Figure \ref{fig:fig3_4}(d)). Decreased conduction with increasing back-gate voltage (ranging from -15 V to 15 V in step sizes of 5 V), was exhibited for both cases.  This decrease in drain-source current against increasing gate bias is expected and indicative of p-type behavior \citep{zhang_p-type_2008, li_structure_2011, liu_fabrication_2013}.  P-type conductivity has often been attributed to Zn vacancies intrinsically present in the ZnTe lattice.  These self-compensation effects have been elaborated on extensively in the discussion of II-VI materials, and are generally ascribed to variations in dissociation pressure for the comprising species that form the final semiconductor crystal, thereby leading to deviation from perfect stoichiometry \citep{marfaing_fundamental_1996}.  In this instance, Zn has a higher vapor pressure than Te, thereby leading to a higher Te content in ZnTe nanostructures, as well as leading to the formation of Te crystillite precipitates, both of which have been present in the growth (Figure \ref{fig:fig4_1}(a))). The defect chemistry and the underlying reaction can be expressed as:

$V_{Zn}{^x} \rightleftharpoons  V_{Zn}{^'} +h^{0}$

$V_{Zn}{^'} \rightleftharpoons V_{Zn}{^''} + h^{0}$

$V_{Zn}{^x} \rightleftharpoons V_{Zn}{^''} + 2h^{0}$

In order to preserve charge neutrality, the excess Te forces Zn$^{2+}$ into a meta-stable trivalent state with a weakly-bound hole.  In the context of the ZnTe band structure, the combination of neutral and ionized Zn vacancies form sets of shallow acceptor levels within the ZnTe bandgap.



The transfer characteristics (I$_{ds}$ vs. V$_{gs}$) of both FETs are also plotted linearly and logarithmically in Figure \ref{fig:fig3_4}(e) and (f).  For nanowires, gate voltages were swept from -40 to 10 V at a drain voltage (V$_{ds}$) of 1.5 V.  Likewise, for nanosheets, gate voltages were swept from -55 to -35 V at the same drain-source bias.  From this analysis, it is found that the on/off ratio of 3.2 $\times$ 10$^{3}$ for nanowires and 3.7 $\times$ 10$^{3}$ for nanosheets.  The threshold voltage (V$_{th}$) was estimated by extrapolating the intersection of the tangent line in the linear region to the V$_{gs}$ axis.  This was estimated to be -8 V for nanowire and -22.75 V nanosheet devices.  
\begin{figure*}[t]
\centering %centers the picture
\includegraphics[scale=0.70]{figures/fig3_5}
\caption{I-V curves indicating increased conduction of (a) nanowire and (b) nanosheet devices at increased power intensities. Maximum current of (c) nanowire and (d) nanosheet devices, as a function of optical power density.  Note: Red line fits data to a power law distribution and a smaller non-unity exponent refers to a greater abundance of traps and recombination centers}
\label{fig:fig3_5}
\end{figure*}





The hole field-effect mobility for both device structures was calculated using the relationship $\mu_{FE}=g_{m}L^{2}/(V_{ds}C_{g})$, where gm is the transconductance measured in the linear region of the transfer curve (g$_{m}$=$d(I_{ds})/d(V_{g})$, L$_{g}$ is the channel length (5 $\mu$m), V$_{ds}$ is the applied drain-source voltage, and C$_{g}$ is the gate capacitance.  The gate capacitance was modeled by the expression C=$2\pi\epsilon_{0}\epsilon_{SiO2}L_{g}/ln(t_{ox}+r)/r)$, where $\epsilon_{0}$ and $\epsilon_{SiO_{2}}$ are the vacuum dielectric constant and the dielectric constant of SiO$_{2}$ (3.9) respectively, $t_{ox}$ is the oxide thickness of the SiO$_{2}$ thermally deposited on the substrate (200 nm), and r is the radius of the nanowire, determined from SEM.  The transconductance, gate capacitance, and field effect mobility were determined to be 1.5 nS, 0.497 fF, and 1.51 cm$^{2}$/V-s, respectively, for nanowires.  The transistor metrics of the nanosheet device employed a slightly altered model with with $\mu_{FE} = g_{m}L/(WV_{ds}C_{0})$ where $C_{0}$ is the capacitance per unit length, and W is the device width (10 $\mu$m).  $C_{0}$ is defined, in this case, as $C_{0}$ = $\epsilon_{0}\epsilon_{SiO2}/t_{ox}$.  This model is typically found in the analysis of nanoribbon and nanobelt devices \citep{li_room_2010, wu_device_2012, wu_construction_2014} due to their planar, rather than cylindrical, surfaces. For nanosheet devices, the transconductance, linear gate capacitance, and field effect mobility were determined to be 212.74 nS, 17.3 nF/cm, and 4.11 cm$^{2}$/V-s.  The carrier concentration for both devices was modeled using the expression n=1/$\rho\mu$q, where $\rho$ is the device resistivity (2.06 $\times$ 10$^{3}$ $\Omega$-cm, 10.56 $\Omega$-cm for nanosheets) and was found to be 2 $\times$ 10$^{15}$ cm$^{-3}$ for nanowires and 1.44 $\times$ 10$^{17}$ cm$^{-3}$ for nanosheets.  \hl{[Not sure about this] Previous reports on bulk ZnTe have also indicated that the occupancy of charge centers can change with illumination, and reduce the effective scattering} charge\citep{fischer_preparation_1964}, \hl{which plays a larger role as the size of the crystal reduces into the nano-scale.  This may explain the discrepancy in field effect mobility between the two nanostructures.}  Lastly, subthreshold swing (S) was analyzed for both device sets by S=ln(10)d(V$_{g}$)/d(ln(I$_{ds}$)).  The subthreshold swing is an important device metric that is a measurement of the devices' switching capability. It was found to be 4.469 V/dec for nanowires and 24.196 V/dec for nanosheets.  A comparison of these device metrics have been outlined against previous reports of ZnTe-based nanostructure devices in Table 1.

From this analysis, it is found that the on/off ratio and field-effect mobility among the two device types are quite comparable, with a slight advantage for nanosheets.  Although nanosheet devices clearly have a much larger transconductance than those composed of nanowires, the gains in mobility are offset by a gate capacitance that is much larger.  This is due to geometric considerations and the overall size of our nanostructures, the W/L ratio of nanosheets far exceeds the inverse of the natural logarithm of the nanowire radius.

\begin{table*}[htbp]
%p{5cm} is the space given to wrap the text
\begin{tabular}{c | p {1 in} | p {1 in} | p {1 in} | p {1 in} }
Device Structure &  \parbox[t]{4cm}{R$_{\lambda}$\\(A/W)} & Photo- \newline conductive Gain & \parbox[t]{4cm}{NEP\\(WHz$^{-.05}$)} & \parbox[t]{4cm}{D$^{*}$\\(cmHz$^{0.5}$W$^{-1}$)}\\
\hline \hline
\parbox[t]{4cm}{ZnTe NW \\ \citep{cao_single-crystalline_2011}} & 2.17 $\times$ 10$^{3}$ & 1.54  $\times$ 10$^{4}$ & n.a. & n.a. &
\parbox[t]{4cm}{ZnTe NW \\ \citep{liu_fabrication_2013}} & 1.87 $\times$ 10$^{5}$ & 4.36 $\times$ 10$^{5}$ & n.a. & n.a. &
\parbox[t]{4cm}{ZnTe NW \\ \citep{li_room_2010}} & 360 & 8.64 $\times$ 10$^{3}$ & n.a. & n.a. &
\parbox[t]{4cm}{ZnTe:Sb nanoribbon \\ \citep{wu_device_2012}} & 4.80 $\times$ 10$^{4}$ & 1.20 $\times$ 10$^{5}$ & 7.5 $\times$ 10$^{-17}$ & 4.8 $\times$ 10$^{13}$ &
\parbox[t]{4cm}{ZnTe NW \\ (this work)} & 2.62 $\times$ 10$^{5}$ & 6.11 $\times$ 10$^{5}$ &  2.99 $\times$ 10$^{-19}$ &  1.10 $\times$ 10$^{16}$ &
\parbox[t]{4cm}{ZnTe nanosheet \\ (this work)} & 110.6 & 257.76 & 9.96 $\times$ 10$^{-17}$ &  2.24 $\times$ 10$^{14}$&
\end{tabular}
\caption{Comparison of ZnTe nanostructure-based photodetector performance metrics}
\label{tab:tbl3_2}
\end{table*}

Under the illumination of a 532 nm green laser, a set of two terminal measurements of both devices structures were also conducted. The drain-source current response to varying light intensities, ranging from 1.7 to 50.24 $\muu$W, is shown in Figure \ref{fig:fig3_5}(a).  As compared to measurements collected under dark conditions, a current increase at the maximum operating voltage (3 V) of ~3x is observed at maximum illumination power (photocurrent increases from 77.2 nA to 1.89 nA) for nanowires, and ~9x at an operating voltage of 2 V for nanosheets.  The inset in each plot shows the response of the maximum photocurrent to power density for both nanostructures.  For semiconductor photoconductors, this relation has typically exhibited a power law dependence\citep{rose_concepts_1963} according to 
I$_{max}$ $\propto$ P$^{\theta}$ . By fitting the experimental photocurrent data to the above expression, we arrive at $\theta$ = 0.2693 for nanowires and $\theta$ = 0.5665 for nanosheets.  

A fractional, non-unity exponent is indicative of a continuous distribution of states (in this case, traps) that convert into recombination centers as the quasi-fermi level shifts due to illumination power.  In our analysis, nanowires have a smaller exponent than nanosheets, which suggests a higher abundance of traps.  The morphology and size configuration of NWs are more surface-dominated than the nanosheet.  As described in previous studies, semiconductor nanowire photodetectors owe their photoconductive properties to surface trap states, and this will becomes more evident upon comparing their photoconductive gains.



The spectral responsivity (R$_{\lambda}$) and photoconductive gain (G) are key figures of merit that distinguish detector systems, and can be defined by the following relationships:

 $\mathlarger{R_{\lambda} =\frac{\Delta I}{PA}}$ and G = $\mathlarger{\left(\frac{\Delta I}{e}\right)}/\mathlarger{\left(\frac{P}{h\nu}\right)}$%(dI/e)/(P/hv)

where $\Delta$I is the difference between the photocurrent and dark (resistive) current, P is the irradiation power, A is the area impinged by photons on the top surface of the nanostructures, $h\nu$ is the bandgap energy, and e is the elementary electronic charge. Theoretically, responsivity can also be expressed in terms of the photoconductive gain, which can also be scaled based on the theoretical quantum efficiency by R$_{\lambda}$ = q$\eta$G/$h\nu$, where $\eta$ is the quantum efficiency defined as $\eta$ = $\eta_{i}$(1 -R$_{ref}$)(1 -e$^{\alpha d}$).  In this expression, $\eta_{i}$ is the intrinsic quantum efficiency (assumed to be 1), d is the absorption depth (diameter for nanowires, thickness for nanosheets), and finally R$_{ref}$ and $\alpha$ are the reflectivity and absorption coefficient of ZnTe (assumed from ZnTe-bulk literature data to be 0.278 and 10^5 cm$^{-1}$), respectively.  In this study, a responsivity of 2.62 $\times$ 10$^{5}$ A/W was calculated for nanowires and 110.59 A/W for nanosheets, at their respective operating voltages.  Accounting for the theoretical quantum efficiency, the photoconductive gains were calculated to be 6.10 $\times$ 10$^{5}$ for nanowires and 257.76 for nanosheets.  Another set of metrics characterizing noise in  photodetectors systems include the noise equivalent power (NEP) and detectivity (D$^{*}$).  These metrics distinguish the sensitivity of detectors, and highly sensitive detectors tend to exhibit low NEP and high D$^{*}$.  Assuming a shot noise-limited relationship (as shot noise is influenced by the devices' dark current, which, in this case, exceeds Johnson noise), they are calculated as NEP = (1/$R_{\lambda}$)(2qI$_{d}$+4kT/$R_{v}$)$^{1/2}$ and D$^{*}$ = (Af)$^{1/2}$/NEP, where k is Boltzmann's constant, T is the absolute ambient temperature, I$_{d}$ is the dark current at the specified operating voltages (18.6 nA for nanowires, 0.352 nA for nanosheets), $R_{v}$ is the device differential resistance (87.6 M$\Omega$ for nanowires, 1.92 G$\Omega$ for nanosheets), and f is the amplifier bandwidth (1 kHz).\citep{sze_physics_2006}  NEP has been calculated to be 2.99 $\times$ 10$^{-19}$ WHz$^{-.05}$ for nanowires and 9.96 $\times$ 10$^{-17}$ WHz$^{-.05}$ for nanosheets, and D$^{*}$ was found to be 1.10 $\times$ 10$^{16}$ cmHz$^{0.5}$W$^{-1}$ for nanowires and 2.24 $\times$ 10$^{14}$ cmHz$^{0.5}$W$^{-1}$ for nanosheets.  Table II compiles detection and noise metrics for a variety of ZnTe nanostructure, thin film and bulk systems, as well as other material systems.

As discussed extensively in ZnO nanowire photodetector studies \citep{soci_zno_2007, yu_zno_2012}, and as has been adapted to ZnTe nanostructure systems by Cao et al., the adsorbtion and desorbtion process of O$_{2}$ along the surface of these materials form traps states from dangling bonds\citep{cao_single-crystalline_2011}.  As the nanostructures are illuminated with a photonic energy greater than the bandgap, this process induces the generation of electron-hole pairs, of which electrons tend to create O$^{-}$ ions along the surface, thereby increase the number of unpaired holes.  In our previous discussion, we found that our ZnTe nanostructures exhibit p-type conduction.  Thus, as the applied voltage across the channel increases, a hole accumulation layer is formed along the surface with reduced band bending.  This results in increased conductivity.  In our analysis, we find that the ZnTe nanowires tend to have a much larger responsivity and photoconductive gain than their nanosheet counterparts.  We can therefore assume that nanowires have a greater number of traps than nanosheets, which is confirmed by our previous analysis of their smaller exponent in the power law relationship relating power flux to photocurrent.  The geometry of our nanowires also demonstrates a larger surface-to-volume ratio as compared to nanosheets, which indicates that the proposed activity along the surface has a much larger impact on the overall conductive response of our devices.  Lastly, we also know that photoconductive gain is traditionally defined as G=$\tau$/$\tau_{tr}$, where $\tau$ is the carrier lifetime and $\tau_{tr}$ is the transit time between contact electrodes.  Hole diffusion is more limited for nanowires, because they have a much more confined active area than nanosheets.  A consequence of this is that holes in nanowires can remain mobile for much longer before they are collected at the electrode or recombine in trap states.  As the difference transit time for both nanostructures is negligible, this could further explain the larger photoconductive gain observed in ZnTe nanowires.  Previous reports which have studied the rise and decay time from illumination in ZnTe nanostructures have confirmed that the size effect has a pronounced impact on carrier lifetime, namely an increase by orders of magnitude \citep{cao_single-crystalline_2011, liu_fabrication_2013}.  Thus, the high responsivity, photoconductive gain, and detectivity make the ZnTe NW a more promising optoelectronic option than the nanosheet.

\section{Conclusion}
ZnTe nanostrucutres, including long  nanowires and wide nanosheets have been synthesized and their performance as field effect transistors and photodetectors have been analyzed. A Schottky barrier with Cr/Au was uncovered, and a barrier height was determined to be 0.48 eV for nanowires and 0.69 for nanosheets.  The devices exhibited hole conduction with comparable  field effect mobilities of 1.51 cm$^{2}$/V-s for nanowires and 4.11 cm$^{2}$/V-s for nanosheets, with nanosheets demonstrating a substantially higher subthreshold swing.  The mechanism for nanosheet formation in the ZnTe material system has been discussed, and the growth related defect chemistry giving rise to p-type behavior has been elaborated upon.  The electrical response of nanostructures were also investigated with the illumination of a 532 nm green diode laser.  Nanowire devices exhibited a responsivity of 2.62 $\times$ 10$^{5}$ A/W, a photoconductive gain of 6.10 $\times$ 10$^{5}$, and a power dependence to light intensity with an exponent of 0.269.  Likewise, the nanosheet responsivity, photoconductive gain, and power distribution exponent were 110.59 A/W, 257.76, and 0.567, respectively.  Additionally, the noise parameters, NEP and D* of the devices have been calculated to be 2.99 $\times$ 10$^{-19}$ WHz$^{-.05}$  and 1.10 $\times$ 10$^{16}$ cmHz$^{0.5}$W$^{-1}$ respectively, for nanowire-based ZnTe devices, and 9.96 $\times$ 10$^{-17}$ WHz$^{-.05}$ and 2.24 $\times$ 10$^{14}$ cmHz$^{0.5}$W$^{-1}$ for nanosheet-based ZnTe devices.  These results indicate that nanowires owe their superior photoconductive performance to their extremely small geometry, leading to a higher surface-to-volume ratio.  This makes the effect of trap states more pronounced for nanowires, leading to longer carrier lifetimes than that nanosheets.