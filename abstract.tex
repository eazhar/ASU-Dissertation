\begin{abstract}
Autonomous smart windows may be integrated with a stack of active components, such as electrochromic devices, to modulate the opacity/transparency by an applied voltage.  Here, we describe the processing and performance of two classes of visibly-transparent photovoltaic materials, namely inorganic (ZnO thin film) and fully organic (PCDTBT:PC$_{70}$BM), for integration with electrochromic stacks.
%Smart windows are integrated with a stack of active components, such as electrochromic devices, whose opacity/transparency can be automatically modulated as a function of externally applied voltage.  However, pairing of photovoltaic and electrochromic devices, which falls within the purview of this dissertation, could lead to autonomous or self-powered smart windows via the utilization of solar energy.  The realization of such an integrated smart window presents various challenges that require tradeoffs among device design, selection of electrical and optical materials, and reproducibility/reliability of low-temperature processing.  Here, we focused on the processing and performance of three classes of visibly-transparent photovoltaic materials, namely, inorganic (ZnO thin film), hybrid organic (PEDOT:PSS/ZnO), and fully organic (PCDTBT:PC$_{70}$BM), for potential integration with electrochromic stacks. 

Sputtered ZnO (2\% Mn) films on ITO, with transparency in the visible range, were used to fabricate metal-semiconductor (MS), metal-insulator-semiconductor (MIS), and p-i-n heterojunction devices, and their photovoltaic conversion under ultraviolet (UV) illumination was evaluated with and without oxygen plasma-treated surface electrodes (Au, Ag, Al, and Ti/Ag).  The MS Schottky parameters were fitted against the generalized Bardeen model to obtain the density of interface states (D$_{it}$ $\approx$ 8.0$\times$10$^{11}$ eV$^{-1}$cm$^{-2}$) and neutral level (E$_{o}$ $\approx$ -5.2 eV).  These devices exhibited photoconductive behavior at $\lambda$ = 365 nm, and low-noise Ag-ZnO detectors exhibited responsivity (R) and photoconductive gain (G) of 1.93$\times$10$^{-4}$ A/W and 6.57$\times$10$^{-4}$, respectively.  Confirmed via matched-pair analysis, post-metallization, oxygen plasma treatment of Ag and Ti/Ag electrodes resulted in increased Schottky barrier heights, which maximized with a 2 nm SiO$_{2}$ electron blocking layer (EBL), coupled with the suppression of recombination at the metal/semiconductor interface and blocking of majority carriers.  For interdigitated devices under monochromatic UV-C illumination, the open-circuit voltage (V$_{oc}$) was 1.2 V and short circuit current density (J$_{sc}$), due to minority carrier tunneling, was 0.68 mA/cm$^2$.

%Inorganic Zinc Oxide (ZnO) thin film-based metal-semiconductor (MS), metal-insulator-semiconductor (MIS), and p-i-n heterojunction devices were investigated, and their photovoltaic conversion under ultraviolet (UV) illumination was evaluated.  For MS devices, photoconductive behavior under ultraviolet illumination ($\lambda$=365 nm), suggesting the outsized role of surface states.  Fitted against the generalized Bardeen model, Schottky parameters were used to estimate the density of interface states (D$_{it}$ $\approx$ 8.0$\times$10$^{11}$ eV$^{-1}$cm$^{-2}$) and the neutral level (E$_{o}$ $\approx$ -5.2 eV).  Post-metalization oxygen plasma treatment of Ag and Ti/Ag electrodes resulted in a net Schottky barrier height increase; linked to the formation of Ag$_{2}$O and TiO$_{x}$. The effective barrier potential maximized with a 20 \AA{} electron blocking layer (EBL, SiO$_{2}$), suppressing recombination at the metal/semiconductor interface and blocking majority carrier current flow. Photovoltaic performance of p-i-n heterojunction structures was maximized generating an open-circuit voltage (V$_{oc}$) of 1.2 V and short circuit current density (J$_{sc}$) of 0.68 mA/cm$^2$ for interdigitated devices under high energy monochromatic UV-C radiation. When properly scaled, ZnO thin film absorbers with sufficiently thin EBL and high surface barrier electrodes are suitable for visibly transparent, low-power smart-windows.

%Poly(3,4-ethylenedioxythiophene) polystyrene sulfonate (PEDOT:PSS) was utilized as a hole generation/transport layer over ZnO as an electron generation/transport layer to augment the power conversion efficiency and photosensitive range of ZnO absorbers.  These devices exhibited photostimulated current-voltage (I-V) hysteresis behavior, in which dissimilar electrical current is observed based on the voltage sweep direction, notably as a function of the illuminated wavelength exposed on the device surface during electrical characterization. Characteristic I-V hysteresis was empirically modeled under a series of first-order multiple linear regression (MLR) expressions that decouple device processing and device characterization conditions.  The results of this analysis indicate that illumination is statistically a stronger explanatory variable for hysteresis than device size, which further suggests that stored space charges on the metal/polymer interface more significantly influence hysteresis than trapped charges alone.


A fully organic bulk heterojunction photovoltaic device, composed of poly[N-9’-heptadecanyl-2,7-carbazole-alt-5,5-(4’,7’-di-2-thienyli2’,1’,3’-benzothiadiazole)]:phenyl-C71-butyric-acidmethyl (PCDTBT:PC$_{70}$BM), with corresponding electron and hole transport layers, i.e., LiF with Al contact and conducting/non-conducting (nc) PEDOT:PSS (with ITO/PET or Ag nanowire/PDMS contacts; the illuminating side), respectively, was developed.  The PCDTBT/PC$_{70}$BM/PEDOT:PSS$_{(nc)}$/ITO/PET stack exhibited the highest performance: power conversion efficiency (PCE) \approx{} 3\%, V$_{oc}$ = 0.9V, and J$_{sc}\approx{}$10-15 mA/cm$^{^2}$.  These stacks exhibited high visible range transparency, and provided the requisite power for a switchable electrochromic stack having an inkjet-printed, optically-active layer of tungsten trioxide (WO$_{3}$), peroxo-tungstic acid dihydrate, and titania (TiO$_{2}$) nano-particle-based blend.  The electrochromic stacks (i.e., PET/ITO/LiClO$_{4}$/WO$_{3}$ on ITO/PET and Ag nanowire/PDMS substrates) exhibited optical switching under external bias from the PV stack (or an electrical outlet), with 7 s coloration time, 8 s bleaching time, and 0.36-0.75 optical modulation at $\lambda$=525 nm.  The devices were paired using an Internet of Things controller that enabled wireless switching.

%A fully organic photovoltaic, composed of poly[N-9’-heptadecanyl-2,7-carbazole-alt-5,5-(4’,7’-di-2-thienyli2’,1’,3’-benzothiadiazole)]:phenyl-C71-butyric-acid-methyl \\(PCDTBT:PC70BM) was explored the final alternative approach toward powering electrochromic devices. The bulk heterojunction exhibited high transparency and relatively large power conversion efficiency and provided the requisite power for transitioning an inkjet printed, nano-particle-based, tungsten trioxide (WO$_{3}$) electrochromic films on substrates of varying mechanical flexibility. The printed electrochromic devices demonstrated clear switching behavior under external bias, with 7 second coloration time, 8 second bleaching time, and 0.36-0.75 optical modulation at $\lambda$=525 nm.   The bulk heterojunction devices were evaluated with varying hole-transport layers and substrates, and exhibited the strongest performance of PCE$\approx{}$3\%, V$_{oc}$=0.9V and J$_{sc}\approx{}$10-15 mA/cm$^{^2}$.  The devices were paired using an Internet of Things controller enabling wireless switching.
\end{abstract}
