\chapter{ZnO Nanowire Based Visible-Transparent Ultraviolet Detectors on Polymer Substrates}
\doublespace
\section{Abstract}
The fabrication and characterization of fully visible-transparent and flexible ultraviolet (UV) detectors, on polyethylene 2,6-naphthalate (PEN) with active channels of zinc oxide (ZnO) nanowires and ohmic indium tin oxide (ITO) contacts, are reported and discussed.  The fabricated detector has an average transmittance of 80\% in the visible spectral range and is most responsive at or below 370 nm, the onset of UV light, with a UV/vis rejection ratio of 1.42 $\times$ 10$^{3}$.  A five orders of magnitude difference in the photocurrent, between UV illumination and dark conditions, is also observed.  The single-sided UV response further shows that the PEN substrate performs well as a UV reflector.  The noise analysis on the nanowire UV detector indicates a noise equivalent power (NEP) and detectivity (D$^{*}$) of 5.88 $\times$ 10$^{-13}$ WHz$^{-0.5}$ and 2.13 $\times$ 10$^{9}$cmHz$^{0.5}$W$^{-1}$, respectively.
\section{Introduction}
In recent years, research effort in the field of flexible and transparent electronics has demonstrated the potential to completely revolutionize consumer products \citep{javey_nano_2006, rogers_materials_2010}.  Unlike the trend in microelectronics, where scaling and performance are the driving force, flexible electronics offer the unique possibility to create integrated devices with multiple functionalities and form factors including implantable, conformable, and multilayer designs with visible transparency.
Wide bandgap UV detectors, especially those based on Zinc Oxide (ZnO), have gained particular momentum \citep{soci_zno_2007, peng_zno_2010, das_zno_2010, law_simple_2006} due to the ease of fabrication on both rigid and opaque substrates.  Most recently, a flexible UV detector based on ZnO thin film has been reported \citep{ji_characteristic_2010}, however no report exists of a visible-transparent, nanowire (NW)-based UV detector on a transparent substrate with both transparent active material and transparent contact material.  An “all-visible” transparent UV detector, as described below, would enable novel applications including the potential to be attached to windows, or atop other devices which require full exposure to visible light.  These devices will eventually further the goals of transparent thin-film electronics, but for now can be used to control automated blinds on smart-windows and support UV photovoltaics.
Zinc Oxide has been explored as an active material for UV detectors because of its large band gap that falls in the UV spectral range (370 nm or 3.3 eV).  Furthermore, ZnO NWs grown from a high temperature process are found to be transparent and mechanically flexible \citep{zhou_flexible_2008}, and their one dimensional formation lend themselves to reduced linear or planar structural defects \citep{martensson_epitaxial_2004}.  Additionally, the likelihood of unwanted trap-state emissions, most notoriously the green band from ZnO thin films and solution-grown ZnO NWs \citep{ozgur_comprehensive_2005, unalan_rapid_2008, kohan_first-principles_2000, liu_growth_2004, vanheusden_mechanisms_1996}, is reduced.  Therefore, these high-quality NWs were used to fabricate fully transparent and flexible UV detector devices on polyethylene 2,6-naphthalate (PEN) substrates with indium tin oxide (ITO) contacts.  The PEN polymer is chosen for its improved resistance to oxidation and its ability to withstand temperatures; higher than comparable polymers such as polyethylene terephthalate (PET) and polytrimethylene terephthalate (PTT) \citep{mackintosh_dynamic_2004}.  Furthermore, PEN has the unique property of exhibiting very low transmittance for shorter UV wavelengths, specifically below 383 nm in comparison to 313 nm for PET \citep{scheirs_modern_2005}.  This makes PEN a suitable candidate for single-sided detection, with high transmittance in the visible range and high rejection in a wider UV range.
\section{Experimental Details}
In this study, the NWs were grown by a typical high temperature Vapor-Liquid-Solid (VLS) process \citep{greene_general_2005}.  Zinc oxide, in high purity powder form, was mixed with graphite in a 1:1 molar ratio and placed at the center of a tube furnace (930 \degree C).  Silicon (100) substrates, with sputtered ZnO thin film (20-30 nm), were placed along the tube at the 700 \degree C growth zone.  The flow rates of 110 sccm for argon and 3 sccm for oxygen were maintained at a steady state pressure of 150 Torr.

%%Figure 1 Defined Here%%
\begin{figure}[t]
\centering %centers the picture
\includegraphics[scale=1]{figures/fig2_1}
\caption{(a) Scanning electron micrograph of ZnO NWs grown in vertical array (inset: top view--scale bar 
1 $\mu$m) and (b) high resolution transmission electron micrograph of ZnO NW indicating highly ordered single crystal structure.}
\label{fig:fig2_1}
\end{figure}
%%Figure 1 Defined Here%%

The Field Emission Scanning Electron Micrograph (FESEM), shown in Figure \ref{fig:fig2_1}(a), exhibits an array of vertically aligned NWs about 3-10 microns in length with hexagonal ends that are about 200 nm in diameter.  Figure \ref{fig:fig2_1}(b) is the high resolution transmission electron micrograph (HRTEM) of a NW; a well ordered single-crystal material free of line defects, with d$_{0001}$ of 0.52 nm are observed.  The absence of deep level defects are further supported by room temperature (295 K) photoluminescence (PL) data, collected with a HeCd laser (325 nm line), in Figure \ref{fig:fig2_2}.  A strong emission due to the band edge transition, with a full width at half maximum (FWHM) of 18 nm, and the absence of dominant broad band peaks in the visible spectral range are clear.  The inset in Figure \ref{fig:fig2_2}, from low temperature PL (12 K), demonstrates a focused narrow band edge emission peak at approximately 369 nm.

%%Figure 2 Defined Here%%
\begin{figure}[h]
\centering %centers the picture
\includegraphics[scale=1]{figures/fig2_2}
\caption{Room temperature photoluminescence (PL) of ZnO NW indicating band edge energy peak and the lack of broad band peaks is attributed to high temperature growth (inset: low temperature PL).}
\label{fig:fig2_2}
\end{figure}
%%Figure 2 Defined Here%%

The NWs were transferred onto the PEN substrate using a mechanical slide setup \citep{fan_wafer-scale_2008}.  The contacts were photolithographically pattered on each end of the NWs and an ITO blanket was deposited using an electron-beam evaporator.  The as-deposited ITO is low density, amorphous, and opaque, but after post-liftoff heat treatment \citep{han_effect_2006} in oxygen, which was optimized to 150 \degree C for 6 hours, the ITO progressively became more transparent as shown in the device of Figure \ref{fig:fig2_3}(a).


\section{Results and Discussion}
The current-voltage characteristics of the two-terminal metal-semiconductor-metal (i.e., ITO-ZnO-ITO) UV detector device are illustrated in Figures \ref{fig:fig2_3}(b) and \ref{fig:fig2_3}(c).  When illuminated with an UV lamp at 365 nm wavelength and power density of 48 mW/cm$^{2}$, the current response at 3 V bias demonstrates a low dark current level of 2 $\times$ 10$^{-11}$ A and a high photoconducting current of 6 $\times$ 10$^{-6}$ A.  The highly linear trace at low bias indicates good ohmic behavior due to the ZnO-annealed ITO contact.  Note, annealing the ITO electrodes resulted in the crystallization and densification, which resulted in the sharp drop in resistivity due to increased mobility \citep{steckl_effect_1980}.  An 80\% transmittance was produced by the annealed ITO contacts, and from four-point probe measurements, a resistivity of 2.85 $\times$ 10$^{-4}$ $\ohm$-cm was determined.  Furthermore, as noted in the logarithmic plot, the photocurrent increases by about 5 orders of magnitude when the single-NW device is illuminated.  Note, at least 20 devices were tested in this manner with most exhibiting comparable characteristics.  In relation to similar ZnO UV detectors, the result of this work is superior to reported ratios of illuminated to dark currents  \citep{soci_zno_2007,peng_zno_2010,ji_characteristic_2010}.

%%Figure 3 Defined Here%%
\begin{figure}[h]
\centering %centers the picture
\includegraphics[scale=1]{figures/fig2_3}
\caption{(a) Fully fabricated visible transparent UV detector after heat treatment (inset: Schematic of nanowire device) (b) IV characteristics under 356 nm UV illumination (black square) and dark conditions (red triangle), Linear plot and (c) Logarithmic plot. Note, the differential resistance of the dark current is found from the inverse slope of the IV characteristics in (b) and is determined to be 1.59$\times$10$^{11} \Omega$}
\label{fig:fig2_3}
\end{figure}
%%Figure 3 Defined Here%%

The spectral photoresponse of the device was determined using a grating monocromator setup (Newport QE/IPCE) that measured the photocurrent at wavelengths from 300 nm to 700 nm.  The ZnO UV detector was operated in air at a bias of 1.5 V, applied through a load matching resistor (100 k$\Omega$) and the photocurrent signal (I$_{ph}$) was measured with a lock-in amplifier.  In addition, a reference photocurrent (I$_{ref}$) was also measured using a detector of known responsivity (R$_{ref}$).  The wavelength dependence of the photoresponse ($\frac{I_{ph}}{I_{ref}}$ x R$_{ref}$) is shown in Figure \ref{fig:fig2_4}.  Note, within the 360-380 nm range, which parallels the FWHM of the PL data, there is a sharp attenuation in the photoresponse.  This result further suggests that the fabricated device performs well as a UV-only detector, as opposed to exhibiting absorption from impurity levels in the visible range that are typical of low-temperature, solution-grown ZnO NW detectors.  Additionally, the rejection ratio of UV to visible light, defined as the ratio of the responsivities at 360 nm and 410 nm, is 1.42 $\times$ 10$^{3}$ for this fully visible-transparent UV detector.

%%Figure 4 Defined Here%%
\begin{figure}[h]
\centering %centers the picture
\includegraphics[scale=1]{figures/fig2_4}
\caption{Measured spectral responsivity of ZnO NW detector at 1.5V bias.  Note, the responsivity at the cutoff wavelength of 360 nm is 2.98$\times$10$^{-7}$ A/W.}
\label{fig:fig2_4}
\end{figure}
%%Figure 4 Defined Here%%

Figure \ref{fig:fig2_5}a illustrates the novel functionality, enabled by the transparent substrate, in which both the front and back (or reverse) side I-V characteristics of the NW device on PEN can be determined.  The schematic for the test setup is shown in the right inset and the results indicate that although PEN is transparent in the visible range, exhibiting an illuminated current level of 56 mA at 1.5 V from the front side, a significantly reduced performance (0.298 mA) from reverse-side illumination occurs due to cutoff transmission wavelength of PEN in the lower energy UV range.  A plot of the device's response to the dark condition is plotted for comparison.  Thus, the magnitude of reduced current from reverse-side illumination may be attributed to the reduced transmittance of PEN; specifically, a reduction in transmittance of three orders of magnitude results in three orders of magnitude difference in the current level between the front and reverse side illuminations.

The photoconductive gain (G) is an important performance metric used to characterize photodetectors.  It is defined as the ratio of the number of electrons collected to the number of photons absorbed per unit time as follows:

\begin{equation}
G = \frac{N_{e^{-}}}{N_{ph}} = \frac{\frac{I_ph}{e}}{\frac{P}{h\nu}}
\end{equation}

where I$_{ph}$ is the photocurrent at the operating voltage, P is the total power impinging on the wire, hν is the bandgap energy of ZnO, and e is the electronic charge.  By geometrically modeling the NW as a cylinder that is exposed on one half, assuming a NW length of 5 $\mu$m and a diameter of 200 nm and with an external light power density of 48 mW/cm$^{2}$, the photoconductive gain at 3 V was calculated to be 4 $\times$ 10$^{5}$.  Note, the value of G is quite comparable with other reports at this specified operating level \citep{soci_zno_2007}; the considerable gain being largely attributed to (a) reduced electron transit time due to the miniaturized dimensions of the active NW channel, and (b) the long carrier lifetime brought upon by ZnO NW surface as explained below \citep{soci_zno_2007}.

%%Figure 5 Defined Here%%
\begin{figure}[b]
\centering %centers the picture
\includegraphics[scale=1]{figures/fig2_5}
\caption{(a) Transmittance of bare PEN on a logarithmic scale (inset: schematic of test setup for illumination) and (b) Current-Voltage characteristics for front (red arrows on top) and reverse (blue arrows under bottom) side illuminations on device compared to dark conditions.}
\label{fig:fig2_5}
\end{figure}
%%Figure 5 Defined Here%%

The use of ZnO NWs for UV detection was reported by Soci et. al. and the mechanism for photoconductive gain has been attributed to the high density of surface trap states; the high surface-to-volume ratio make these trap states a dominant factor.  The trap states apparently stem from adsorbed oxygen molecules that capture free electrons under no illumination or illumination with sub-band gap energy of ZnO.  The consequent formation of an interface depletion layer with low surface conductivity leads to the suppressed conductivity under dark conditions.  Under UV illumination, electrons and holes are created within the ZnO NW.  The photogenerated holes readily migrate to the surface to neutralize the charged oxygen molecules.  Consequently, the free electrons, with high carrier lifetimes, directly contribute to the high conductivity within the depletion-free NW, and can be efficiently collected at the electrode.

The fabrication of UV detectors on PEN serves as a good application on a window, for example, when UV detection coupled with the simultaneous blocking of harmful UV energy from transmission needs to take place.  Moreover, exploration of other polymer substrates, with shorter cutoff wavelengths, could enable detection on both sides if required.  Reports of NW devices fabricated on PET \citep{zhang_high-performance_2009} would allow for photodetection of wavelengths at least 70 nm below PEN and devices fabricated on PTT \citep{zhang_determination_2004} would even permit another 60 nm below PET.  The current work on fully visible-transparent application and previous studies on using ZnO NWs for UV detection may enable a variety of innovative design architectures in the future.

Finally, two figure of merits characterizing the noise of the fabricated photodetector, namely the noise equivalent power (NEP) and detectivity (D$^{*}$), were also analyzed at room temperature.  It is widely known that defects in ZnO (i.e., Zn interstitials and O vacancies) give rise to the observed n-type behavior \citep{banerjee_recent_2005, xiang_rational_2007, cao_phosphorus_2007}; the device’s dark current is evidence of free electrons.  Shot noise is a consequence of the dark current of the nanowire detector, and has a noise magnitude of 5.83 $\times$ 10$^{-13}$ WHz$^{-0.5}$, which exceeds both the Johnson noise (7.66 $\times$ 10$^{-14}$ WHz$^{-0.5}$) and 1/f noise at the measured amplifier bandwidth of 1 kHz.  Thus, since the thermally-limited model may not be applicable, NEP and D* are evaluated using the following relations\citep{jiang_high-speed_2007}: NEP = ($\frac{1}{R_{\lambda}}$)*(2qI$_{d}$+4kT/R$_{\nu}$)$^{1/2}$ and D$^{*}$ = (A*f)$^{1/2}$, where R$_{\lambda}$ is the responsivity at the selected detection wavelength of 360 nm, q is the elementary electronic charge, I$_{d}$ is the dark current at the detection bias of 1.5 V, R$_{v}$ is the device differential resistance, k is the Boltzmann constant, and A is the device area, and f is the amplifier bandwidth \citep{sze_physics_2006}.  Note, R$_{v}$ and I$_{d}$ from Figure \ref{fig:fig2_3} are 1.59 $\times$ 10$^{11}$ $\Omega$ and 9.42 $\times$ 10$^{-12}$ A, respectively.  For an exposed surface area (A) of 1.57 $\times$ 10$^{-13}$ m$^{2}$ and R$_{\lambda}$ of 2.98 $\times$ 10$^{-7}$ A/W, the room-temperature NEP is found to be 5.88 $\times$ 10$^{-13}$ WHz$^{-0.5}$ and the corresponding D$^{*}$ is 2.13 $\times$ 10$^{9}$ cmHz$^{0.5}$W$^{-1}$.  The NEP of this work is comparable to thin film UV detectors; this is attributed to the low dark current, which is also an indication of high quality ZnO nanowires.  Typically, thin-film ZnO UV detectors exhibit higher D$^{*}$ compared to NW-based detectors; for unprotected single-nanowire devices, the lower D$^{*}$ may be attributed to more pronounced carrier trapping and detrapping effects, stemming from a large surface-to-volume ratio, and consequently a much smaller effective device area \citep{lu_noise_2007}.  Here, a summary of various literature reports for comparison with the current data is given in Table 1.

\begin{table*}[t]
%p{5cm} is the space given to wrap the text
\begin{tabular}{p{1 in} | p{1 in} | p{0.75 in} | p{0.75 in} | p{0.75 in} | p{1 in}}
ZnO material & Illuminated/ \newline dark current (@ 3 V) & Rejection \newline ratio (UV/vis) & Photo-\newline conductive gain & NEP \newline (WHz$^{-0.5}$)  & Detectivity \newline (cmHz$^{0.5}$W$^{-1}$)\\
\hline \hline
Nanowire \newline \citep{soci_zno_2007} & 7.5$\times$10$^{4}$ & 4.3 $\times$ 10$^{1}$ (air) & 2 $\times$ 10$^{7}$ & NR & NR &
Thin film \newline \citep{ji_characteristic_2010} & 1.61$\times$10$^{3}$ & 1.56 $\times$ 10$^{3}$ & NR & NR & NR &
Nanowire \newline \citep{lu_noise_2007} & 20 & 10 & NR & 7.89$\times$10$^{-11}$ & 1.9$\times$10$^{8}$ @2V &
Thin film \newline \citep{young_zno_2007} & 10$^{4}$ & ~10$^{3}$ & NR & 3.17$\times$10$^{-13}$ & 2.23$\times$10$^{12}$@1V &
Thin film \newline \citep{jiang_high-speed_2007} & NR & 5$\times$10$^{5}$ & NR & NR & 1.37$\times$10$^{11}$@3V &
Nanowire (this work) & 10$^{5}$ & 1.42 $\times$ 10$^{3}$ & 4 $\times$ 10$^{5}$ & 5.88$\times$10$^{-13}$ & 2.13$\times$10$^{9}$@1.5V &
\end{tabular}
\\
\caption{Performance Metrics of ZnO UV Detectors}
\label{tab:tbl2_1}
\end{table*}

%Insert Table Here
%References: [3 7 27 28 29]
%References: [
%3:soci_zno_2007 
%7: ji_characteristic_2010  
%27: lu_noise_2007
%28:young_zno_2007 
%29: jiang_high-speed_2007]

\section{Conclusion}
Fully visible-transparent and flexible UV detectors have been fabricated and characterized by using high quality ZnO NWs and ITO electrodes on a PEN substrate.  The photoconductive properties on the resultant devices have been thoroughly examined, which indicate a five orders of magnitude difference in the photocurrent difference between UV illumination and dark conditions.  In addition, the rejection ratio of the NW devices' responsivity to both the UV and visible spectrum was found to be 1.47 $\times$ 10$^{3}$, which is quite large and comparable to related reports.  The photocurrent response of the ZnO NW detector to the reverse-side exposure of light was studied and found to be a function of the cutoff transmittance frequency of PEN.  Combined with the substantial photoconductive gain (G=4 $\times$ 10$^{5}$), the fabricated device performs well as a single-sided and fully visible-transparent UV detector.  The noise analysis of the nanowire UV detector indicates a noise equivalent power (NEP) and detectivity (D$^{*}$) of 5.88 $\times$ 10$^{-13}$ WHz$^{-0.5}$ and 2.13 $\times$ 10$^{9}$cmHz$^{0.5}$W$^{-1}$, respectively.  A comparison of noise parameters, in detectors employing ZnO in various geometries, shows the efficacy and potential of NW based devices.
\section{Acknowledgments}
The authors acknowledge the support of the National Science Foundation-ECCS (0926017), the Arizona State University Center for Solid State Electronic Research (CSSER) for processing facilities, the Center for Solid State Science (CSSS) for NW growth facilities, and the Flexible Display Center (FDC) for use of polymer substrates.  We thank Luying Li, Martha McCartney and David Smith for assistance with HRTEM and Christian Poweleit for assistance with PL measurements.