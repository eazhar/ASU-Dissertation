\chapter{Vapor-Transport Synthesis and Annealing Study of Zn$_{x}$Mg$_{1-x}$O Nanowire Arrays for Selective, Solar-Blind UV-C Detection}
\section{Abstract}
This work uniquely reports the synthesis of Zn$_{x}$Mg$_{1-x}$O nanowires and submicron columns by utilizing a traditional carbothermal reduction process toward forming ZnO nanowire ultraviolet detectors, while simultaneously utilizing Mg$_{3}$N$_{2}$ as the source of Mg. To investigate the relationship between Mg content in the ZnO lattice and the cutoff wavelength for high spectral responsivity, the nanowires were annealed in a series of designed conditions, while chemical, nanostructural, and optoelectronic characteristics were compared before and after treatment.  Post-anneal scanning electron micrographs revealed a reduction of the average ensemble nanowire dimensions, which was correlated to the modification of ZnO lattice parameters stemmin from Zn$^{2+}$ dissociation and Mg$^{2+}$ substitution (confirmed via Raman spectroscopy).  Analysis of cathodoluminescence spectra revealed a blue-shift of the peak alloy band edge emission along with a red-shift of the ZnO band edge emission; and both were found to be strong functions of annealing temperature.  The conversion of Zn$_{2}$SiO$_{4}$ to Mg$_{2}$SiO$_{4}$ (in O$_{2}$) and MgSiO$_{3}$ (in Ar), was found to correspond to transformations (shifting and scaling) of high energy luminescence peaks, and was confirmed with XRD analysis.  The tunability of the cutoff photodetection wavelength was evaluated as the nanowire arrays exhibited selective absorption by retaining elevated conduction under high-energy UV-C irradiation after thermal treatment, but exhibiting suppressed conductivity and a single order of magnitude reduction in both spectral responsivity (R$_{\lambda}$) and photoconductive gain (G) under UV-A illumination.  Noise analysis revealed that the variation of detectivity (D$^{*}$) depended on the regime of ultraviolet irradiation, and that these variations are related to thermal noise resulting from oxygen-related defects on both nanowire and substrate surfaces.  These results suggest a minor design tradeoff between the noise characteristics of thermally treated ZnMgO nanowire array UV detectors and the tunability of their spectral sensitivity.

\section{Introduction}
Ultraviolet, solar-blind communication systems that exploit atmospheric scattering to propagate signals toward a non-line of sight (NLOS) receiver (with ranges on the order of kilometers) have been examined extensively, yet the detectors in these receiver systems have largely been dominated by bulky and costly photomultiplier tubes (PMT) \citep{chen_experimental_2008, el-shimy_binary-input_2012, yang_algorithm_2014}.  Semiconductor-based deep-UV detectors have consequently become of great interest due to their potential advantages of producing low-cost, low-power-consumption, highly scalable solutions.  While ZnO nanowire-based photodetectors have been heavily investigated, this material system only exhibits a cutoff detection corresponding to the band edge energy of ZnO in the UV-A spectrum (3.10-3.94 eV) \citep{ting_li_low-noise_2001, collins_improved_2002, cui_solar-blind_2016}.  Substitutional doping of Mg with ZnO, on the other hand, has been described as a means of engineering the bandgap of a ternary Zn$_{x}$Mg$_{1-x}$O system as high as a 5.8 eV for solar blind photodetectors \citep{tang_characterizations_2010, liu_heteroepitaxial_2010, lange_mgzno_zno_2011, vanjaria_broad_2016}.  Although Mg has been reported as a feasible dopant to ZnO (due to the similar ionic radii of Zn$^{2+}$ and Mg$^{2+}$) \citep{hwang_effects_2004, liu_deposition_2005}, Mg/ZnO alloy systems have also shown to phase segregate for high Mg content \citep{huso_low_2007}.  Thus, the development of reliable techniques toward synthesizing aligned ZnO nanowires, and appreciably incorporating Mg during synthesis is of great significance.  Prior work investigating ZnMgO nanostructures have been reported based on deposition methods, which have ranged from: molecular beam epitaxy \citep{pietrzyk_growth_2014}, metal-organic chemical vapor deposition \citep{kim_density_2011,thierry_coreshell_2012}, pulsed-laser deposition \citep{polyakov_shallow_2009}, hydrothermal techniques \citep{liu_heteroepitaxial_2010, shimpi_low_2009}, RF magnetron co-sputtering \citep{kar_fabrication_2008}, and vapor-phase transport \citep{zhou_morphology_2009, tang_characterizations_2010, vanjaria_broad_2016}. The vapor-phase transport technique, yields highly crystalline nanostructures and is a relatively straightforward synthesis route, able to encompass equilibrium formation in one step, rather than necessitating \textit{ex-situ} incorporation.  Additionally, understanding the changes in optical and physical properties due to post-growth thermal treatment of ZnMgO nanowires is of importance toward tuning the cutoff wavelength for solar-blind photodetectors.

In this study, ZnMgO nanowires and submicron columns were synthesized on Si via two equilibrium processes occurring simultaneously: (1) the carbothermal reduction of ZnO and (2) the incorporation of Mg from dissociated Mg$_{3}$N$_{2}$, which to the best knowledge of the authors has not been reported. Furthermore, a controlled study varying annealing environment and annealing temperature was conducted, demonstrating an evolution of the chemical, spectral, and optoelectronic properties of the synthesized ZnMgO nanostructures.  Assessed both before and after thermal treatments, these modified characteristics were investigated via Field Emission Scanning Electron Microscopy (FESEM), Energy-dispersive X-ray spectroscopy (EDX), X-ray diffraction (XRD), Raman spectroscopy, cathodoluminescence (CL), and current-voltage (I-V) behavior.  The structural changes of ZnMgO nanowires manifested as modifications of aspect ratio, while also correlating to modified estimates of Mg incorporation due to each annealing condition.  Analysis of spectral responsivity and photoconductive gain between treatment and control groups was conducted under dissimilar ultraviolet regimes to numerically compare the differences within detector performance.  By exploiting these controlled reaction kinetics from thermal annealing, enhanced selective filtering of high energy UV-C absorption on the nanowire ensemble is demonstrated, accompanied by suppressed conductivity under UV-A illumination. Additionally, analysis of detector noise characteristics demonstrated that their variation from ultraviolet irradiation was linked to the increase of thermal noise from increased oxygen-related defects, presenting potential performance tradeoffs.

\begin{figure}[t]
\centering %centers the picture
\includegraphics[scale=0.68]{figures/fig6_6}
\caption{Schematic of reaction mechanisms forming ZnMgO nanowires, along with structural and chemical modifications from thermal treatment.}
\label{fig:fig6_6}
\end{figure}

\begin{table*}[t]
%p{5cm} is the space given to wrap the text
\begin{tabular}{p{0.8 in}  p{1.3 in} || p{0.9 in}  p{0.9 in} || p{0.5 in}  p{0.5 in}}
Anneal \newline Environment&Anneal \newline Temperature (\degree{}C)&2$\theta$ Peak \newline(100) Plane&2$\theta$ Peak \newline(002) Plane&c (\AA{})&a (\AA{})\\
\hline \hline
Control & Control & 31.71708287 & 34.39533709 & 5.21 & 3.25 &
Ar & 650 & 31.86748429 & 34.40787054 & 5.21 & 3.24 &
Ar & 900 & 31.96775190 & 34.52067161 & 5.19 & 3.23 &
O$_{2}$ & 650 & 31.86748429 & 34.50813816 & 5.19 & 3.24 &
O$_{2}$ & 900 & 31.99281881 & 34.62093923 & 5.18 & 3.23 &
\end{tabular}
\\
\caption{Lattice Parameter modification of ZnMgO wires as a function of annealing condition}
\label{tab:tbl6_1}
\end{table*}


\section{Experimental Details}
ZnMgO nanowires were synthesized using a vapor phase transport method comprised of a carbothermal reduction of ZnO in tandem with the dissociation of Mg$_{3}$N$_{2}$ as the source of Mg.  In a single zone reaction tube furnace, a ceramic boat containing a 1:1 molar ratio mixture of ZnO and graphite powder (total 3 g) was placed upstream to another ceramic boat containing 10 g of Mg$_{3}$N$_{2}$ powder. Silicon (100) substrates were cleaned in piranha solution (70\% H$_{2}$SO$_{4}$, 30\% H$_{2}$O$_{2}$) and were treated using a buffered oxide etchant (BOE, 2\% HF). Subsequently, a thin 10 nm ZnO film was sputtered (Lesker PVD 250) atop substrates to form a catalyst seed.  The samples were then cleaned using acetone, isopropanol, and water before being positioned downsteam relative to the source materials.  The tube furnace was evacuated and brought to 500 \degree C.  Upon reaching the target temperature, Ar was introduced to the system at a flow rate of 87.4 sccm, while maintaining a steady state pressure of 150 Torr throughout the process.  At a rate of 500 \degree C/hour, the furnace temperature continued increasing up to 1100 \degree C, upon which the temperature was held for 24 minutes.  In the final 10 minutes of this growth period, O$_{2}$ was introduced to the system at a flow rate of 7.5 sccm. Finally, the system was allowed to cool naturally, without gas flow, under vacuum.  This process produced a conformal, coarse, light-gray deposition consisting of nanowire and submicron columnar growth on the substrate surface.  The sample was then cleaved into 5 equally sized pieces, four of which were sealed in separate quartz ampoules of Ar and O$_{2}$ environments. The sealed samples underwent a unique combination of thermal annealing (650 \degree{}C and 900 \degree{}C), with a fixed anneal time of 30 minutes (conditions summarized in Table 1).  In all cases, a reflective metallic reduced Zn deposit (confirmed with EDX) was found to have condensed along the ampoule walls.  The four samples, along with the unnannealed "control" case, were morphologically characterized with a field emission scanning electron microscope (FESEM, Hitachi S4700-II, excitation 15 kV). Post-growth analysis of nanowire size distributions was conducted by tracing radial and axial dimensions of a sample size of approximately n=1000 nanowires, for each treatment group, along 9 equally spaced sample regions (corners, edges, center) within imageJ (image processing) software.  Relative Mg content was determined using an Energy-dispersive X-ray spectroscopy (EDX) attachment by sampling the aforementioned regions along each sample surface, at an excitation energy of 15 kV. Modification of crystal structures was determined with a High Resolution X-ray Diffractometer (PANalytical XPert PRO XRD, $\kappa\alpha_{1}$=1.540598 \AA{}, $\kappa\alpha_{2}$=1.544426 \AA{}).  Ultraviolet luminescence spectra were ascertained using a JEOL JSM 630 SEM with an attached cathodoluminescence (CL) system, operating at an excitation current of 0.5 nA for UV-A analysis and 2 nA for UV-C analysis.  Raman spectra were gathered with a Renishaw InVia spectroscopy system with a 100$\times$ objective lens, using a laser source of $\lambda$ = 488 nm.  Finally, conductance modification subject to UV illumination (UVP EL Series Lamp, 45 mW/cm$^{2}$) were determined with a Kiethly 4200 Semiconductor Analyzer System within a "light-tight" micromanipulator probe-station, utilizing tungsten-tipped soft probes.  Post-experimental processing of data which included cleaning, analysis, and visualization were performed in R language with the ggplot2 library.


\section{Results and Discussion}

\begin{figure}[t]
\centering %centers the picture
\includegraphics[scale=0.64]{figures/fig6_1}
\caption{(a) SEM Micrograph of ZnMgO nanowire array (inset: zoomed view of single wire) (b) EDX Mapping of grown nanowires, uncovering spatial resolution of Zn, Mg, and O$_{2}$ constituents (c) Left: Measured size distributions (top: diameter, bottom: length) of nanowires for each anneal treatment group. Right: sample micrograph of traced nanowire lengths, post-anneal. (d) Elemental atomic ratio as a function of anneal temperature and anneal gas}
\label{fig:fig6_1}
\end{figure}

Figure \ref{fig:fig6_1}(a) shows the scanning electron micrograph (SEM) of the as-synthesized ZnMgO nanowire arrays, with a zoomed view of a single nanowire in the inset.  The wires exhibited a vertical (yet regularly offset by a slight angle) growth pattern with a characteristic coarse surface texture.  Energy-dispersive X-ray spectroscopy (EDX) mapping in Figure \ref{fig:fig6_1}(b) further revealed the elemental composition of the nanowires, confirming the incorporation of Mg.  As seen in Figure \ref{fig:fig6_1}(d), EDX quantification analysis shows that the Mg content relative to Zn extends as high as 15\% upon the initial synthesis.  A growth mechanism that accounts for the carbothermal reduction of ZnO with a Mg$_{3}$N$_{2}$ source yields reaction products of Zn$_{x}$Mg$_{(1-x)}$O, ZnO, MgO, and Zn$_{2}$SiO$_{4}$, as illustrated in Figure \ref{fig:fig6_6}.  Upon subjecting the nanowires to each anneal condition, the average relative composition of Mg in all anneal cases widens in variance, and is most pronounced when annealed in O$_{2}$ at 900 \degree C.  In this case, the median Mg content relative to Zn is found to be approximately 24\%, as shown in Figure \ref{fig:fig6_1}(d).  Similarly reported behavior has indicated that the alloying Mg content reaches saturation due to phase segregation, despite larger annealing temperatures \citep{liu_heteroepitaxial_2010}. From Figure \ref{fig:fig6_1}(c) the effect of annealing on nanowire size (diameter and length) does not ostensibly appear pronounced, however an analysis of variance (ANOVA) coupled with a pairwise Tukey's range test confirms a statistically significant reduction in the distributions of both nanowire dimensions, especially for higher temperature (see supporting information Table S.1 and S.2).  This reduction is explained by a modification of the underlying ZnO nanostructure, with Zn$^{2+}$ vaporizing and dissociating from the nanowire \citep{lamoreaux1987high}, allowing Mg$^{2+}$ to occupy the vacancies and integrate within the crystal lattice.  As presented in Figure \ref{fig:fig6_1}(d), this process is evidenced by a decrease of relative Zn:O content for all nanowire arrays, while Mg content is found to increased relative to Zn, post-anneal.  In a similar study, Kim et al. describe coating MgO nanowire surfaces with particle-like ZnO crystallites that sinter upon annealing, giving rise to increased dimensions \citep{kim_effects_2007}, and was further attributed to an increase in oxygen vacancies that allowed for the relaxation of interfacial strain \citep{shimpi_annealing_2010}. In our work, we find that the ZnO lattice consumes Mg$^{2+}$, and nanowire dimensions decrease as the atomic exchange kinetics are reversed.

\begin{figure}[t]
\centering %centers the picture
\includegraphics[scale=0.53]{figures/fig6_2}
\caption{(a) Compiled XRD Spectra of as-synthesized nanowires (labeled as control) and all nanowire anneal conditions. (b) Zoomed view of ZnO peaks emerging from the (100) plane and (c) ZnO (002) plane}
\label{fig:fig6_2}
\end{figure}

Presented in Figure \ref{fig:fig6_2}(a) are the x-ray diffractograms (XRD) of all annealed ZnMgO nanostructure arrays, along with the as-synthesized case (labeled as \textit{control}).  From the diffraction pattern of the as-synthesized nanowires, characteristic peaks associated with ZnO, Zn$_{2}$SiO$_{4}$, and MgO (JCPDS 05-0664, JCPDS 00-024-1469, and JCPDS 04-0849, respectively) can be observed (which indicates a slight degree of phase segregation between wurtzite ZnO and cubic MgO phases).  Because the nanostructures retain hexagonal form, cubic MgO is not as predominant a reaction product as ZnO. Additionally, the formation of Zn$_{2}$SiO$_{4}$ is largely confined to the substrate, as confirmed by EDX mapping. As shown in Figures \ref{fig:fig6_2}(b) and \ref{fig:fig6_2}(c), annealing results in a right-shift of the ZnO a-axis corresponding to the (100) plane (2$\theta$ = 31.7\degree{}), as well as the ZnO c-axis corresponding to the (002) plane (2$\theta$ = 34.4\degree{}), by as high as $\Delta$2$\theta$ = 0.3\degree{} in both cases (Note: the presence of bimodal peaks arises from dissimilar $\kappa\alpha$ excitation energies emitted from the diffractometer).  Principally, the combined effect of a smaller electronegativity and electronic radius of Mg$^{2+}$ (Pauling electronegativity: 1.31 and 0.57 \AA{} radius) to that of Zn$^{2+}$ (Pauling electronegativity: 1.65 and 0.60 \AA{} radius) facilitates the formation of the Mg---O bond and leads to the decrease of observed lattice parameters \citep{liu_deposition_2005, singh_anomalous_2011, wei_annealing_2012, das_effect_2013, saha_effect_2015}.  Thus, Mg$^{2+}$ is effectively substituting Zn$^{2+}$ without significantly modifying the ZnO nanowire crystal structure, which further explains the reduced nanowire size distributions from Figure \ref{fig:fig6_1}(c). However, when these modified lattice parameters are scaled up to the average nanowire dimensions, we find that this phenomenon alone does not completely account for the observed reduction in array aspect ratio. Instead, reduced radial and axial dimensions also stem from Zn$^{2+}$ dissociating completely from the nanowires, which is especially evident at high temperatures; and is supported by Zn deposits found along the sealed ampoule walls.  Nonetheless, a summary of the 2$\theta$ peak shifts and lattice parameters is presented in Table \ref{tab:tbl6_1}. In addition to shifts of ZnO related peaks, new peaks emerge due to reactions on the substrate surface when annealed at high temperatures.  Annealing in inert Ar results in the formation of MgSiO$_{3}$ (enstatite, JCPDS 01-088-1924), whereas annealing in reactive O$_{2}$ results in the formation of Mg$_{2}$SiO$_{4}$ (forsterite, JCPDS 01-079-1490).  Notably, the diffraction pattern and luminescence spectra of these mineral forms are a strong function of crystal structure and imperfect stochiometry, and these results are further discussed in the context of cathodoluminescence analysis.

\begin{figure}[t]
\centering %centers the picture
\includegraphics[scale=0.75]{figures/fig6_3}
\caption{(a) Compiled Raman Spectra of control group and all anneal conditions (b) Extracted 1LO Phonon shift extracted from Raman spectra as a function of temperature}
\label{fig:fig6_3}
\end{figure}

The presence of Mg$^{2+}$ behaving as a substitutional ion in a fundamentally ZnO nanostructure is suported by the Raman spectrogram shown in Figure \ref{fig:fig6_3}(a).  Increasing temperature shifts the peak position of the 1LO phonon mode from approximately 585 cm$^{-1}$ to 605 cm$^{-1}$, as summarized in Figure \ref{fig:fig6_3}(b). The modification of the 1LO peak position is found to be a greater function of temperature than that of the ambient gas environment, and is attributed to temperature-sensitive contributions from the LO mode of MgO \citep{huso_phonon_2014} at 720 cm$^{-1}$.  Prior studies have reported that the maximal Mg content that the ZnO crystal can accommodate is 30\%, corresponding to a Raman shift \citep{huso_phonon_2015} of 615 cm$^{-1}$.  A model that maps Raman shift to Mg content was developed by Huso et al. utilizing fine control of Mg in sputtered ZnO thin films to explain a bowing feature in the Raman spectral response \citep{ye_effects_2007}.  By cross-referencing the change in 1LO peak position to this model, as well as the modification of 2$\theta$ peak diffractions of ZnO, the estimated Mg content incorporated is demonstrably increased via annealing, and agrees with EDX analysis.



\begin{figure}[t]
\centering %centers the picture
\includegraphics[scale=0.45]{figures/fig6_4}
\caption{(a.1) Raw cathodoluminescence spectra at low-energy ultraviolet range (UV-A) near the ZnO band edge with (a.2) normalization of peak intensity and extracted peak intensities of (a.3) ZnMgO alloy and (a.4) ZnO.  (b.1) Cathodoluminescence Spectra at higher-energy ultraviolet range (UV-C) highlighting  (b.2) MgSiO$_{3}$ and Mg$_{2}$SiO$_{4}$ formation from 170-190 nm peak shift and (b.3) Zn$_{2}$SiO$_{4}$ dissociation at 230 nm with reduced peak intensity as a function of temperature and annealing gas environment.}
\label{fig:fig6_4}
\end{figure}

The cathodoluminescence (CL) spectra of all ZnMgO wires were measured both in the UV-A range, presented in Figure \ref{fig:fig6_4}(a.1), and in the UV-C range, presented in Figure \ref{fig:fig6_4}(b.1).  While the relative intensity of the band edge of ZnO ($\approx$380 nm) dominates all other peaks across the spectrum, the intensity of the peak is found to decrease with increasing annealing temperature, given the same level of CL excitation current.  This is attributed to the increased formation of O$_{2}$ defects along the wires, leading to an increased green-band emission (see supporting information Figure S.1).  The higher energy UV-C cathodoluminescence response, as presented in Figure \ref{fig:fig6_4}(b.1), was normalized to the peak at 156 nm, corresponding to the bandgap of MgO.  From the normalized response, the relative intensities of two other peaks can be compared: specifically those between 165-183 nm (linked to the band energy ranges of Mg$_{2}$SiO$_{4}$ and MgSiO$_{3}$) \citep{shankland_band_1968, stashans_modelling_2010} and at 230 nm (linked to the band energy of Zn$_{2}$SiO$_{4}$) \citep{mishra_first_1991, karazhanov_electronic_2009}.  The progressive attenuation of the Zn$_{2}$SiO$_{4}$ peak intensity, shown in Figure \ref{fig:fig6_4}(b.3), and the increase of the Mg$_{2}$SiO$_{4}$ and MgSiO$_{3}$ peak intensity, shown in Figure \ref{fig:fig6_4}(b.2), illustrate a relative exchange in composition, directly resulting from dissimilar anneal environments.  Because molecules of Zn tend toward vapor pressures lower than that of Mg, ZnO nanostructures and Zn$_{2}$SiO$_{4}$ will more preferably dissociate into Zn$^{2+}$, O$_{2}$, and SiO$_{4}^{4-}$ at higher anneal temperatures \citep{lamoreaux1987high}.  Isolated orthosilicate ions must therefore react to form a forsterite molecules with Mg, or equilibrate into silica if reaction-limited. As previously discussed, Zn deposits were found along the sealed ampoules, which indicate a vaporization and dissociation of Zn$^{2+}$ from the nanowire arrays.  As O$_{2}$ is a reactive gas, Zn$_{2}$SiO$_{4}$ is found to have completely converted into Mg$_{2}$SiO$_{4}$ \citep{shankland_band_1968}.  This is validated by the fact that the relative CL peaks associated with Mg$_{2}$SiO$_{4}$ rise considerably under an oxidizing environment, until forming a close shoulder to the MgO band edge (in other words, exhibiting a distinct species, rather than coalescencing with MgO). Annealing in Ar reduces the relative peak intensity of the Zn$_{2}$SiO$_{4}$ peak to approximately 75\% of its original level, and in the inert environment, must only react with MgO in a kinetically-limited manner to form half as much (molar ratio) of MgSiO$_{3}$ \citep{stashans_modelling_2010} (see reactions of both anneal environments in Figure \ref{fig:fig6_6}).  Mishra et al. have described a modification of the orbital symmetries t$_{1}$ and t$_{2}$ of SiO$_{4}^{4-}$ ions, which have been influenced by the oxidizing reactions mechanics, contributing to a range of orthosilicate ion peak position shifts found in Figure \ref{fig:fig6_4}(b.1).  These results are illustrated in Figure \ref{fig:fig6_6}, and summarize the complex reaction kinetics of Zn$^{2+}$ dissociation in conjunction with silicate formation as a function of the annealing environment.

\begin{figure}[t]
\centering %centers the picture
\includegraphics[scale=0.5]{figures/fig6_5}
\caption{Current-Voltage measurement of ZnMgO NW photodetector as a function of illumination conditions for nanowires that were (a) initially grown and (b) received thermal treatment at 900 \degree{}C in O$_{2}$}
\label{fig:fig6_5}
\end{figure}



Current-Voltage measurements were conducted vertically across the nanowire arrays for the as-synthesized case, shown in Figure \ref{fig:fig6_5}(a), and for samples annealed at 900 \degree{}C in O$_{2}$, shown in Figure \ref{fig:fig6_5}(b).  Measurements were conducted in the dark and in the presence of  UV-A (365 nm) and UV-C (254 nm) monochromatic illumination.  The photoconductive properties of ZnO nanowires have long been attributed to the size effect of nanostructures, in correspondence with the adsorption/desorption of O$_{2}$ molecules along the nanowire surface leading to surface band bending and reducing channel resistivity \citep{soci_zno_2007, yu_zno_2012, xie_enhanced_2014}.  The control case exhibited a negligible difference in conductance regardless of UV illumination wavelength, which suggests that band-to-band transition along the ZnO band edge largely contributes to its photoconductive behavior.  However, in the annealed case, UV-A photoconduction is suppressed due to the increased formation of Zn vacancies, as evidenced by the attenuation of the peak ZnO band-edge luminescence, from in Figure \ref{fig:fig6_4}(a.1).  Specifically, the conductance was found to increase by almost 2 orders of magnitude (126 times) when illuminated with a UV-C source, as compared to a lower energy UV-A source, in which the annealed nanowire array only exhibited an increase of conduction by a factor of 11.  

\begin{equation} \label{eq:1}
R_{\lambda} =\frac{\Delta I}{PA}
\end{equation}

\begin{equation} \label{eq:2}
G = \mathlarger{\left(\frac{\Delta I}{e}\right)}/\mathlarger{\left(\frac{P}{h\nu}\right)}
\end{equation}



The spectral responsivity (R$_{\lambda}$) and photoconductive gain (G) are key figures of merit that distinguish photodetector systems. These figures are defined by the relationships expressed in eq. (\ref{eq:1}) and (\ref{eq:2}), where $\Delta$I represents the difference between the photocurrent and dark current, P is the irradiation power, A is the area impinged by photons on the top surface of the nanostructures, $h\nu$ is the illumination energy, and e is the elementary electronic charge \citep{yu_zno_2012}.  The results of both performance characteristics are summarized in Table \ref{tab:tbl6_2} for treated and untreated cases, compared between UV-A and UV-C illumination.  In agreement with the observed changes in photoconduction, the most notable differences in R$_{\lambda}$ and G occur for UV-A illumination ($\lambda$=365 nm), in which at least a single order of magnitude difference is estimated between the as-grown (R$_{\lambda} _{| \lambda=365}$ = 1.70$\times 10^{-5}$ (A/W), G$_{|\lambda=365}$ = 5.79$\times 10^{-5}$) and annealed cases (R$_{\lambda} _{| \lambda=365}$ = 1.78$\times 10^{-4}$ (A/W), G$_{|\lambda=365}$ = 9.85$\times 10^{-4}$), illustrating a suppression of UV-A photoconduction in the nanowire ensemble.  Likewise, these figures do not change appreciably with UV-C illumination.  

\begin{equation} \label{eq:3}
NEP = (1/R_{\lambda})(2qI_{d}+4kT/R_{v})^{1/2}
\end{equation}

\begin{equation} \label{eq:4}
D^{*} = (Af)^{1/2}/NEP
\end{equation}


\begin{table*}[t]
%p{5cm} is the space given to wrap the text
\centering
\begin{tabular}{r || c c}
Photodetection Metric & As-Synthesized & Annealed\\
\hline \hline
R$_{\lambda} _{| \lambda=365}$ (A/W) & 1.70$\times 10^{-5}$ & 1.78$\times 10^{-4}$ &
R$_{\lambda} _{| \lambda=254}$  (A/W) & 2.20$\times 10^{-4}$ & 2.15$\times 10^{-4}$ &
\hline
G$_{|\lambda=365}$ & 5.79$\times 10^{-5}$ & 9.85$\times 10^{-4}$ &
G$_{|\lambda=254}$ & 9.47$\times 10^{-4}$ & 1.05$\times 10^{-3}$ &
\hline
NEP$_{|\lambda=365}$ (WHz$^{-0.5}$)& 1.02$\times 10^{-8}$ & 1.56$\times 10^{-9}$ &
NEP$_{|\lambda=254}$ (WHz$^{-0.5}$) & 7.91$\times 10^{-10}$ & 1.46$\times 10^{-9}$ &
\hline
D*$_{|\lambda=365}$ (cmHz$^{0.5}$W$^{-1}$) & 3.10$\times 10^{9}$ & 2.03$\times 10^{10}$ &
D*$_{|\lambda=254}$ (cmHz$^{0.5}$W$^{-1}$) & 4.00$\times 10^{10}$ & 2.17$\times 10^{10}$ &
\end{tabular}
\\
\caption{Comparison of photodetection performance and noise metrics between as-grown and thermally treated ZnMgO nanowires as a function of ultraviolet illumination energy}
\label{tab:tbl6_2}
\end{table*}


To further investigate detection performance disparities, detector noise characteristics were calculated in order to distinguish the sensitivity of ZnMgO nanowires before and after thermal anneal.  Noise equivalent power (NEP) and detectivity (D$^{*}$) were evaluated according to eq. (\ref{eq:3}) and (\ref{eq:4}), where k is Boltzmann's constant, T is the absolute ambient temperature, I$_{d}$ is the dark current for a specified operating voltages, $R_{v}$ is the device differential resistance, and f is the amplifier bandwidth (taken as 1 kHz in this analysis) \citep{sze_physics_2006}.  Along with detector performance characteristics, detector noise metrics are also summarized in Table \ref{tab:tbl6_2}, and demonstrate a reduction of NEP for UV-A illumination (due to decreased spectral sensitivity), whereas a slight increase in NEP is observed for UV-C illumination.  As highly sensitive detectors tend to exhibit low NEP and high D$^{*}$, the sensitivity of ZnMgO detectors increases in conjunction with suppressed photoconductive gain at $\lambda$=365 nm.  Because annealing was found to reduce Zn content from the nanowire system, electrons from ZnO become less effective at band-to-band transitions, which diminishes the effect of surface band-bending from oxygen desporption \citep{feng2012handbook}.  On the substrate surface, the formation of Mg$_{2}$SiO$_{4}$ results in stochiometricly heterogeneous MgO films, thereby increasing thermal noise.  Annealing in oxygen was shown to increase the incidence of O$_{2}$ related green-band defects (see supporting information Figure S.1), which contribute directly to thermal agitation of charge carriers \citep{hsu_origin_2004, djurisic_defect_2007, lee_zno-based_2015}.  However, this NEP increase is negligible in comparison to the modification of spectral sensitivity under UV-A illumination, which suggests a minor design tradeoff between noise and tunability.

\section{Conclusion}
In conclusion, high-quality ZnMgO nanowires have been synthesized utilizing a novel growth mechanism that incorporates Mg dissociated from Mg$_{3}$N$_{2}$, in tandem with the carbothermal reduction of ZnO. Modification of the size characteristics, crystal lattice properties, and spectral response from annealing were found to be internally consistent with measured Mg content. In particular, EDX, Raman, and CL revealed a spectral evolution for each treatment group of nanowire arrays, giving rise to increased Mg$^{2+}$ incorporation, along with the dissociation of Zn$^{2+}$. Statistical inference (ANOVA paired with Tukey's Range Test) was undertaken to discern a reduction of nanowire dimensions, and was cross-referenced to a modification of the ZnO lattice parameters, as well as the formation of Zn vacancies.  

Suppressed photoconductivity under UV-A illumination with retained photoconductivitiy under high energy UV-C illumination was observed for samples post-thermal treatment. An order of magnitude difference for both spectral responsivity and photoconductive gain between the annealed and control cases confirms the tunable filtering of high energy ultraviolet spectral responsivity.  Increased thermal noise from the high temperature O$_{2}$ anneal process was associated with the increase of oxygen-related defects contributing to thermal agitation, which may present a design tradeoff in integrating ZnMgO as solar blind detectors.  Thus, these results indicate that synthesized ZnMgO nanowires demonstrate promising selective UV-C detection capabilities, in which controlled post-synthesis thermal treatment may effectively tune for selective, high-pass, solar-blind ultraviolet detection.
